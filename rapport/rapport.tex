
%----------------------------------------------------------------------------------------
%	PACKAGES AND OTHER DOCUMENT CONFIGURATIONS
%----------------------------------------------------------------------------------------

\documentclass[11pt]{article}

\usepackage{fancyhdr} % Required for custom headers
\usepackage{lastpage} % Required to determine the last page for the footer
\usepackage{extramarks} % Required for headers and footers
\usepackage[usenames,dvipsnames]{color} % Required for custom colors
\usepackage{graphicx} % Required to insert images
\usepackage{listings} % Required for insertion of code
\usepackage{courier} % Required for the courier font
\usepackage{array,multirow,makecell}
\usepackage[utf8]{inputenc}
\usepackage{indentfirst} %Indentation début de paragraphe
\usepackage{float}
\usepackage{colortbl} %Clouleur tableau protoypes de fonctions
\usepackage{alltt}
\usepackage{fancyvrb}
\usepackage{hyperref}
\usepackage{amsmath}

% Marges
\topmargin=-0.45in
\evensidemargin=0in
\oddsidemargin=0in
\textwidth=6.5in
\textheight=9.0in
\headsep=0.25in

\linespread{1.1} % Line spacing

%  Mise en place des pieds de page et en-têtes.
\pagestyle{fancy}
\lhead{QuokkaLight} % Top left header
\chead{Ducky the LKM Rootkit} % Top center head
\rhead{Master CSI} % Top right header
\lfoot{\lastxmark} % Bottom left footer
\cfoot{} % Bottom center footer
\rfoot{Page\ \thepage\ sur\ \protect\pageref{LastPage}} % Bottom right footer
\renewcommand\headrulewidth{0.4pt} % Size of the header rule
\renewcommand\footrulewidth{0.4pt} % Size of the footer rule

%\setlength\parindent{10pt} % Removes all indentation from paragraphs

%----------------------------------------------------------------------------------------
%	Page de Titre
%----------------------------------------------------------------------------------------

\title{
\pagenumbering{roman} \setcounter{page}{0} %La page courante sera numérotée en roman et aura l'indice 0 => Pas de numéro car pas de 0 en roman
\vspace{2in}
\textmd{\textbf{\hmwkClass:\ \hmwkTitle}}\\
\normalsize\vspace{0.1in}\small{\hmwkDueDate}\\
\vspace{0.1in}\large{\textit{\hmwkClassInstructor\ }}
\vspace{3in}
}

\author{\textbf{\hmwkAuthorName}}

%----------------------------------------------------------------------------------------
%	Page de garde
%----------------------------------------------------------------------------------------


\makeatletter
\def\clap#1{\hbox to 0pt{\hss #1\hss}}%
\def\ligne#1{%
\hbox to \hsize{%
\vbox{\centering #1}}}%
\def\haut#1#2#3{%
\hbox to \hsize{%
\rlap{\vtop{\raggedright #1}}%
\hss
\clap{\vtop{\centering #2}}%
\hss
\llap{\vtop{\raggedleft #3}}}}%
\def\bas#1#2#3{%
\hbox to \hsize{%
\rlap{\vbox{\raggedright #1}}%
\hss
\clap{\vbox{\centering #2}}%
\hss
\llap{\vbox{\raggedleft #3}}}}%
\def\maketitle{%
\thispagestyle{empty}\vbox to \vsize{%
\haut{}{\@blurb}{}
\vfill
\vspace{1cm}
\begin{flushleft}
\usefont{OT1}{ptm}{m}{n}
\huge \@title
\end{flushleft}
\par
\hrule height 1pt
\par
\begin{flushright}
\usefont{OT1}{phv}{m}{n}
\Large \@author
\par
\end{flushright}
\vspace{1cm}
\vfill
\vfill
\bas{}{\@location, le 21 Mars 2016}{}
}%
\cleardoublepage
}
\def\date#1{\def\@date{#1}}
\def\author#1{\def\@author{#1}}
\def\title#1{\def\@title{#1}}
\def\location#1{\def\@location{#1}}
\def\blurb#1{\def\@blurb{#1}}
\date{\today}
\author{}
\title{}
\location{Bordeaux}\blurb{}
\makeatother
\title{Programmation d'un LKM rootkit}
\author{Thomas Le Bourlot, Maxime Peterlin, Martial Puygrenier}
\location{Bordeaux}
\blurb{
Université de Bordeaux\\
Faculté de Sciences et Techniques\\
Ducky the LKM Rootkit
}%

%----------------------------------------------------------------------------------------

\begin{document}

\thispagestyle{empty}
\maketitle
\newpage

%----------------------------------------------------------------------------------------
%	Table des matières
%----------------------------------------------------------------------------------------

%\setcounter{tocdepth}{1} % Uncomment this line if you don't want subsections listed in the ToC
\pagenumbering{arabic} \setcounter{page}{1} 
\thispagestyle{empty}
\renewcommand\contentsname{Sommaire}
\tableofcontents
\newpage

%----------------------------------------------------------------------------------------
%	Partie à remplir
%----------------------------------------------------------------------------------------

\newcommand*{\escape}[1]{\texttt{\textbackslash#1}}
\newcommand*{\escapeI}[1]{\texttt{\expandafter\string\csname #1\endcsname}}
\newcommand*{\escapeII}[1]{\texttt{\char`\\#1}}

\section{Introduction}

Lorem

\section{Rootkits  - quésaco ?}

Lorem

\section{Injection - Comment le rootkit est injecté en mémoire ?}

Lorem

	\subsection{Méthode : /dev/kmem}
		\subsubsection{Explication}
	
			Lorem
		
		\subsubsection{Contre-mesures}
	
			Lorem

	\subsection{Méthode : LKM}
		\subsubsection{Explication}
			Lorem
		\subsubsection{Contre-mesures}
			Lorem
	
\section{Détournement de l'exécution du noyau}	

	\subsection{Détournement des appels systèmes}
	Lorem
	\subsection{ Détournement du VFS (Virtual File System)}
	Lorem
\section{Persistance du rootkit}
	Lorem
\section{Fonctionnalités du rootkit}
	Lorem
\section{Conclusion}
	Lorem
\section{Annexes}
	Lorem
\section{Bibliographie}
	Lorem
\end{document}